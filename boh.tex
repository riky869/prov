\documentclass[10pt]{article}

\usepackage[utf8]{inputenc}
\usepackage{geometry}
\usepackage{tabularx}
\usepackage{graphicx}

\graphicspath{{images/}}

%cambio misure della pagina
\geometry{a4paper,left=20mm,right=20mm,top=20mm}

\title{Valutazione Capitolati}
\date{A.A 2024/2025}

\renewcommand*\contentsname{Indice}
\begin{document}
%contenuti principali
\maketitle
\begin{center}
\includegraphics[width=0.25\textwidth]{LogoUnipd}\\
\includegraphics[width=0.25\textwidth]{Sevenbitslogo}\\
sevenbits.swe.unipd@gmail.com\\
\vspace{2mm}

\textbf{Registro modifiche}\\
\vspace{2mm}
\begin{tabular}{|l|l|l|l|l|l|}
\hline
\textbf{Versione} & \textbf{Data} & \textbf{Descrizione} & \textbf{Ruolo} & \textbf{Componente} \\
\hline
1.0 & 29/10 & Stesura e revisione del preventivo & Scrittore & Gusella Manuel\\
\hline
\end{tabular}
\end{center}
\tableofcontents
\newpage
\section{Valutazione del capitolato scelto}
\subsection{Capitolato C4: "NearYou - Smart custom advertising platform"}
\textbf{Descrizione:}
\begin{itemize}
    \item Proponente: Synclab srl
    \item Committenti: Prof. Tullio Vardanega e Prof. Riccardo Cardin
    \item Obiettivo: 
\end{itemize}
\textbf{Dominio applicativo:}
\textbf{Dominio tecnologico:}
\textbf{Motivazioni della scelta:}
\textbf{Conclusioni:}

\section{Valutazioni sui capitolati rimanenti}
\subsection{Capitolato C1: “Artificial QI” }
\begin{itemize}
    \item Proponente: Zucchetti Spa
    \item Committenti: Prof. Tullio Vardanega e Prof. Riccardo Cardin
    \item Obiettivo: Realizzare un sistema che valuti la capacità dei modelli AI di rispondere a domande complesse, mettendo a confronto risposte attese con quelle generate dai modelli. Lo scopo è creare uno strumento che possa testare e migliorare le performance dei modelli AI attraverso metodi di valutazione flessibili e configurabili.
\end{itemize}
\textbf{Dominio applicativo:}
Il progetto si inserisce nel contesto dei sistemi basati su modelli di Intelligenza Artificiale, con particolare attenzione alla valutazione e alla validazione delle risposte generate nei sistemi che usano LLM.
\textbf{Dominio tecnologico:}
Utilizzo di Large Language Models (LLM) e sviluppo di API Rest per l’integrazione con sistemi esterni. Nella documentazione non vengono indicate tecnologie specifiche: è stato confermato durante il colloquio con l’azienda che tutte le scelte tecniche sono a carico del gruppo, sulla base degli strumenti che si padroneggiano già.
\textbf{Aspetti positivi:}
\begin{itemize}
    \item Progetto innovativo, in linea con le tendenze AI e LLM
    \item Libertà di esplorare tecnologie e approcci diversi
    \item Opportunità di approfondire competenze avanzate di testing e valutazione con AI
    \item Si distingue dalla proposta del classico chatbot, frequente fra gli altri capitolati
\end{itemize}
\textbf{Aspetti negativi:}
\begin{itemize}
    \item Libertà eccessiva nelle scelte progettuali e tecnologiche
    \item Mancanza di linee guida chiare e carico decisionale elevato
    \item Non negli interessi complessivi del gruppo
\end{itemize}
\textbf{Conclusioni:}
Il capitolato C1 rappresenta un’opportunità formativa interessante e orientata alla sperimentazione su AI e LLM. Tuttavia, il progetto è stato scartato per la libertà eccessiva nella scelta delle tecnologie e delle metodologie di implementazione, dato che il gruppo tende a preferire indicazioni più precise, per avere una maggiore chiarezza operativa e progettuale.

\subsection{Capitolato C2 “Vimar GENIALE”}

\subsection{Capitolato C3 “Automatizzare le routine digitali tramite l’intelligenza generativa”}

\subsection{Capitolato C5 “3Dataviz”}

\subsection{Capitolato C6 “Sistema di Gestione di un Magazzino Distribuito”}

\subsection{Capitolato C7 “LLM: Assistente virtuale”}

\subsection{Capitolato C8 “Requirement Tracker - Plug-in VS Code”}

\subsection{Capitolato C9 “Progetto BuddyBot”}

\end{document}

